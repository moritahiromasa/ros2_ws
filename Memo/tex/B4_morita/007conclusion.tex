%#! platex 000main.tex
\chapter{結論}
  \label{chap:conclusion}
  本手法では, 端点を用いて人のような描き方ができるのかを調査した. 線を辿るように経路を生成したので,逆に一筆書きのようになった. 実際に多くの生成した経路が目や鼻,耳などから頭上へ繋がる下から上へ写る描かれ方をしていて,不自然な書き方になっていた.
  また最も近い端点に移動すれば, 近似的に線を描けると仮定したが, 途切れたエッジの補間や交点を考慮して経路を求めなかったため,望んだものとは少々違う軌道になることが多かった.
  全体としてラスタスキャンに方が優位であったが,生成結果は上から下へとプロッターのように単純な描き方であった.
  まとめると,線は上部から下部へと移り,それに加えて本手法のように線が繋がって描かれていると良いと考えられる.

