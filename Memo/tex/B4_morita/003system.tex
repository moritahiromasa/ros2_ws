%#!platex 000main.tex
\chapter{システム構成}
  \label{chap:system}
  \section{ハードウェア}
  	\label{chap:hardware}
 	ロボットの機体はAdeeptのADA031を用いている. Arduinoで動作するため、画像処理の難易度が高い. そこでメインボードをRaspberry Piに代え、それとロボットアームの各servoモータを接続するためにブレッドボードやジャンパー線、抵抗を用いて改良を行った.
  \section{ロボットの機構}
  	\label{chap:mechanism}
	本研究で用いるロボットは3軸のマニピュレータロボットである. servo1を原点とし、各servoの回転角度を$\theta_1, \theta_2, \theta_3$、各リンクの長さを$l_1, l_2, l_3, l_4$とする.
  
  \section{手先の位置と回転角の関係}
  	\label{chap:kinetic}
	リンク座標系に定義した各パラメータと、順運動学と逆運動学を用いて手先の位置から回転角を求める. またDH(Denavit-Hartenberg)記法を用いて、各リンクの関係を示す.
	このリンク座標系をもとに、座標系1から3への座標変換行列を求めると以下のようになる.
	\begin{equation}
		\boldsymbol{ ^{0}T_{3}=^{0}T_{1} ^{1}T_{2} ^{2}T_{3} }
		\begin{array}{cc}
			=
			\left(
				\begin{array}{cccc}
					C_1C_{23} & -C_1S_{23} & 0 & l_2C_1S_2 \\
					S_1C_{23} & -S_1S_{23} & 0 & l_2S_1S_2 \\
					-S{23} & -C_{23} & -1 & l_2C_2 + l_1 \\
					0 & 0 & 0 & 1
				\end{array}
			\right)
		\end{array}
	\end{equation}
	ただし、
	\begin{equation*}
		\left(
		\begin{split}
			C_{23} = cos\theta_2cos\theta_3-sin\theta_2sin\theta_3\\
			S_{23} = sin\theta_2cos\theta_3+cos\theta_2sin\theta_3
		\end{split}
		\right)
	\end{equation*}
    また、手先ベクトルが以下のように求まる.
	\begin{equation}
	\begin{array}{cccc}
		&\left(
			\begin{array}{cccc}
				1 & 0 & 0 & l_4\\
				0 & 1 & 0 & 0\\
				0 & 0 & 1 & 0\\
				0 & 0 & 0 & 1
			\end{array}
		\right)
		\left(
			\begin{array}{cccc}
				1 & 0 & 0 & 0\\
				0 & 1 & 0 & -l_3\\
				0 & 0 & 1 & 0\\
				0 & 0 & 0 & 1
			\end{array}
			\right)
		&=
		\left(
		\begin{array}{cccc}
			1 & 0 & 0 & l_4\\
			0 & 1 & 0 & -l_3\\
			0 & 0 & 1 & 0\\
			0 & 0 & 0 & 1
		\end{array}
		\right)
	\end{array}
	\end{equation}
	式(3.1)と(3.2)より手先までの座標変換行列$\boldsymbol{ ^{0}P_{r} }$が以下のように求まる.
	\begin{equation*}
		\boldsymbol{ ^{0}P_{r} } =
		\boldsymbol{^{0}T_{3}}
		\left(
		\begin{array}{cccc}
			1 & 0 & 0 & l_4\\
			0 & 1 & 0 & -l_3\\
			0 & 0 & 1 & 0\\
			0 & 0 & 0 & 1
		\end{array}
		\right)=
		\begin{array}{cc}
			\left(
				\begin{array}{cc}
					\boldsymbol{R} & \boldsymbol{t} \\
					\boldsymbol{0} & 1
				\end{array}
			\right)
		\end{array}
	\end{equation*}
	ただし、回転行列$\boldsymbol{R}$と並進$\boldsymbol{t}$は以下の結果になる.
	\begin{equation*}
	\begin{split}
		&\boldsymbol{R} =
		\begin{array}{cc}
			\left(
				\begin{array}{ccc}
					C_1C_{23} & -C_1S_{23} & 0 \\
					S_1C_{23} & -S_1S_{23} & 0 \\
					-S{23} & -C_{23} & -1
				\end{array}
			\right)
		\end{array} \\
		&\boldsymbol{t} =
		\begin{array}{cc}
			\left(
				\begin{array}{c}
					l_4C_1C_{23}+l_3C_1S_{23}+l_2C_1S_2 \\
					l_4S_1C_{23}+l_3S_1S_{23}+l_2S_1S_2 \\
				-l_4S_{23}+l_3C_{23}+l_2C_2+l_1
				\end{array}
			\right)
		\end{array}\\
		&\boldsymbol{0} =
		\begin{array}{cc}
			\left(
				\begin{array}{ccc}
					0 & 0 & 0
				\end{array}
			\right)
		\end{array}
	\end{split}
	\end{equation*}

	座標変換行列の並進ベクトル$\boldsymbol{t}$が順運動学解なので、手先の位置x, y, zが以下のように求まる.
	\begin{equation}
			\begin{array}{c}
			\begin{split}
				& x = C_1(l_4C_{23} + l_3S_{23} + l_2S_2) \\
				& y = S_1(l_4C_{23} + l_3S_{23} + l_2S_2) \\
				& z - l_1 = -l_4S_{23} + l_3C_{23} + l_2C_2 \\
			\end{split}
		\end{array}
	\end{equation}
	式(3.3)より逆運動学解$\theta_1, \theta_2, \theta_3$は以下のように求まる.
	\begin{equation}
			\begin{array}{c}
			\begin{split}
				\theta_1  &  =\frac{1}{2}  cos^{-1}\biggl( \frac{x^2-y^2}{x^2+y^2} \biggr) \\
				\theta_2 & = cos^{-1}\biggl( \frac{x^2+y^2+(z-l1)^2 + l_2^2-l_3^2-l_4^2}{2l_2\sqrt{x^2+y^2+(z-l_1)^2}} \biggr) + tan^{-1}\biggl( \frac{\sqrt{x^2+y^2}}{z-l1}\biggr) \\
				\theta_3 & =cos^{-1}\biggl( \frac{x^2+y^2+(z-l1)^2 - l_4^2-l_3^2-l_2^2}{2l_2\sqrt{l_3^2+l_4^2}}\biggr) + tan^{-1}\biggl( \frac{-l_4}{l_3}\biggr)
			\end{split}
			\end{array}
	\end{equation}

