%#! platex 000main.tex
\chapter{関連研究}
  \label{chap:related}
  
  描画ロボットに関する研究と、線画を求めるためにエッジをできるだけ残せるようにする必要がある.
	
  \section{描画ロボットに関する研究}
    \label{sec:related_reasearch}
	文献\cite{1}では、6軸のマニピュレータロボットを用いて、シンプルにエッジ抽出から人物画を描く研究を行っている.
	また文献\cite{2}では、Cannyやラプラシアンなど様々なエッジ抽出と細線化、そして影の部分などをハッチングで表現できるように、芸術的な人物画を鉛筆で描く研究を行っている. 
	またリモートユーザがタブレットを介してロボットに描かせる文献\cite{3}などが存在する.


  \section{エッジ抽出に関する研究}
    \label{sec:edged_detection_research}
	用途にもよるがエッジ抽出において、Cannyのエッジ検出アルゴリズムが最もよく用いられる.
	エッジ抽出を行うとき、しきい値を高くするとノイズは少なくなるが、エッジが見えなくなってしまう. 逆にしきい値を低くするとエッジは残るが、ノイズがはっきりと現れてしまう. これらのトレードオフ関係に取り組んだ論文が文献\cite{4}である. この文献は画像を領域ごとに仕切り分けを行い、その領域ごとにエッジがあるのか、ノイズがあるのか、両方あるのかを判定し、領域ごとにしきい値の値を決め、トレードオフ関係に対処しようとしている. 結果的に改良に用いたCannyのエッジ検出アルゴリズムより若干向上したことを示している.



