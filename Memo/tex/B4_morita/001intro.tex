%#! platex 000main.tex
\chapter{序論}
  \label{chap:intro}
  \section{背景}
    \label{sec:background}
	従来の描画ロボットの研究では描かれた絵が、どのぐらい上手く描けているかに着目している研究が多い. 例えばシンプルにエッジ抽出から人物画を描く文献\cite{1}や、エッジ抽出とハッチングから芸術的な人物画を鉛筆で描く文献\cite{2}、リモートユーザがタブレットを介してロボットに描かせる文献\cite{3}などが存在する. これらの研究では模写や芸術的な表現が可能になっているが、人のような描き方を追求したものが少ないと感じた.
  \section{本研究の目的}
    \label{sec:target}
	風景や静物などは描き手によってかなり描き順に差があるが、人物画の頭部であればある程度パターンがあると考えた.  加えて、ヒトの頭部は表情が存在するため絵に様々な意味を連想させたり、静止した一場面にストーリーをもたらせたる、背景などの他の要素をより際立たせることができる. これは絵画以外の漫画にも当てはまることで、頭部を描く重要性は高い.	そこで本研究では、ある画家が描いた頭部作品の画像から描き順ができるだけ人に近い描き方をする描画ロボットを作成を行う.\\
	

	\section{本論文の構成}
    \label{sec:construction_of_this_paper}
	最初に関連研究について述べる. 次にシステムの構成について述べる. ここでは用いたハードウェアや手先の位置から回転角を導出するまでの過程や、線画を生成するために行った画像処理について述べる. その次に提案手法を述べる. ここでは人のような描き方ができるようにある領域の用意と端点の検出を行い、経路を求めた. 最後にラスタスキャンと領域、端点を用いて場合とでどちらが人のような描き方に見えるか比較を行った.
    
