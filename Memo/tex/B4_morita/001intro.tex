%#! platex 000main.tex
\chapter{序論}
  \label{chap:intro}
  \section{背景}
    \label{sec:background}
	従来の描画ロボットの研究では描かれた絵が、どのぐらい上手く描けているかに着目している研究が多い.
    例えばシンプルにエッジ抽出から人物画を描く文献\cite{1}や、エッジ抽出とハッチングから芸術的な人物画を鉛筆で描く文献\cite{2}、リモートユーザがタブレットを介してロボットに描かせる文献\cite{3}などが存在する.
	これらの研究ではたしかに模写や芸術的な表現が可能になっている.
	\\
	\hspace{10pt}古来よりヒトは頭部を数多く描いてきた.
	文字を描く技術が生まれ、絵を技術が生まれ、記録することができるようになった.
	しかし映像として記録できるようになるまでの間、ヒトは文章や絵で記録を行う.
	絵を描くことで当時の災害や争いの悲惨さを伝えたり、カリスマ性のある指導者の偉大さを強く表現したりした.
	またヒトの頭部も多く描かれた.
	なかでもヒトの頭部は表情が存在するため絵に様々な意味を連想させたり、静止した一場面にストーリーをもたらせたり、背景などの他の要素をより際立たさせる.
	静止しているにも関わらず、ストーリーをもつ点では映像とはまた別の独特な魅力を持っている.
	これは絵画以外に漫画などにも当てはまることで、頭部を描く重要性は高い.\\
	\hspace{10pt}現代ではディジタルで絵を描いたり、AIに絵を生成してもらうことが当たり前になっている.
	絵の場合、アナログの複製はほとんど意味をなさない. 
	たとえ見た目が同じに見えても、絵の具を使った量や、混ぜた色の比率、筆のタッチの刻み方を完全に再現できることは不可能であるからだ.
	それだけ、優れた絵描きの絵のオリジナリティは強く評価される.
	また時間と労力は有限であるため、作品数は限られる.



  \section{本研究の目的}
    \label{sec:target}
	従来の描画ロボットの研究を調査した結果、人のような描き方を追求したものが少ないと感じた.
	そこで本研究では、ある画家が描いた頭部作品の画像から描き順ができるだけ人に近い描き方をする描画ロボットを作成を行う.
	\hspace{10pt}
   現代において鉛筆デッサンをする場合、線画を描く、つまり輪郭をはっきり描く方法は一般的ではない.
	これは古典的なデッサンの方法で昔の巨匠、例えばミケランジェロやダビンチなどが輪郭を描いている.
	しかし全く行われないというわけでもなく、画家によっては輪郭を描く.
	また、輪郭を描くことで何かしら特別な意味を込めたい場合などに描く場合もある. 
	
	\section{本論文の構成}
    \label{sec:construction_of_this_paper}
    
